\documentclass[9pt]{article}
\usepackage[a4paper,margin=25mm]{geometry}
\usepackage{graphicx}
\usepackage{longtable}
\usepackage{xcolor, colortbl}
\usepackage{sectsty}
\usepackage{tikz}
\usepackage{titlesec}
%\usepackage[default,osfigures,scale=0.95]{opensans}
%\renewcommand\familydefault{\sfdefault}
\usepackage[T1]{fontenc}
\titleformat*{\section}{\Large\sffamily\color[HTML]{012E13}}
\sectionfont{\color[HTML]{012E13}}
\usepackage{fancyhdr}
\pagestyle{fancy}
\lhead{\color[HTML]{012E13}\scriptsize CS 11: COMPUTER PROGRAMMING I}
\rhead{\color[HTML]{012E13}\scriptsize SYLLABUS 17-18A}
\cfoot{\color[HTML]{012E13}\scriptsize \thepage}
\renewcommand{\headrulewidth}{0.2pt}
\renewcommand{\footrulewidth}{0.2pt}

\begin{document}\thispagestyle{empty}
\arrayrulecolor{gray}
\begin{center}
	\includegraphics[width=1in]{UPD-DCS.png}\\\vspace{10pt}
	University of the Philippines, Diliman\\
	College of Engineering\\
	Department of Computer Science
\end{center}

\section*{Course Information}
\begin{center}
	\begin{tabular}{p{2in} p{3.5in}}
		Course Number: & CS 11\\
		Course Schedule: & \textbf{WFWX} W 1:00-4:00 lec, F 1:00-4:00 lab\newline\textbf{WFUV} W 10:00 - 1:00 lec, F 10:00 - 1:00 lab\\
		Course Title:	& Computer Programming I\\
		Course Description: & Introduction to computer science. Problem-solving strategies. Algorithm development. Coding conventions.
Debugging. Fundamental programming constructs: types, control structures, functions, I/O. Basic data structures.\\
		Credit: & 3 units\\
		Co-requisite: & Math 17\\
		Methodology: & In-class lectures, programming exercises and projects\\
		Instructor:	& Kristofer E. delas Pe\~nas\\
		Consultation: & M 1:00AM-4:00PM, WF 10:00-1:00PM, Rm. 307, UPAECH\\
		Email Address: & kedelaspenas@up.edu.ph\\
	\end{tabular}
\end{center}
\section*{Course Goals}
\paragraph{}
At the end of the course, the student should be able to:
\begin{itemize}
	\item 
\end{itemize}
\section*{Course Outline}
\begin{enumerate}
	\item 
\end{enumerate}
\section*{Course Requirements}
\section*{Class Policies}
\begin{itemize}
	\item \textbf{Collaboration}\\
	Students are allowed to informally collaborate on their assignments, exercises and machine problems with other students who have taken the course previously or are currently taking the course. Submitting code copied verbatim or nearly verbatim even with proper citation is prohibited unless otherwise specified by the instructor.
	
	\item \textbf{Consultation}\\
	Consultation is encouraged. A student who wants to consult should inform the instructor at least a day before his/her preferred day of consultation.
	
	\item \textbf{Loss of Work}\\
	Students should make backup copies of all their work in this course. Loss of work due to hardware failure will not be considered as an acceptable excuse for late submission or non-submission of requirements.  
	
	\item \textbf{Previous Work}\\
	Students are free to use programs they have written in the past provided they follow the required format and are authorized by the instructor.
	
	\item \textbf{Cheating}\\
	Any instance of copying the works and/or thoughts of others and passing it as one's own is considered as plagiarism. In using course materials, students should be careful not to claim words, ideas and algorithms as one's own.
	
\end{itemize}
\section*{Grading System}
	\paragraph{}
	Students will be graded according to the following scale:
	\begin{center}
		\begin{tabular}{|p{2in}|p{2in}|}
			\hline \textbf{General Average} & \textbf{Final Grade}\\
			\hline [92 - 100] & 1.0\\
			\hline [88 - 92) & 1.25\\
			\hline [84 - 88) & 1.50\\
			\hline[80 - 84) & 1.75\\
			\hline [76 - 80) & 2.00\\
			\hline [72 - 76) & 2.25\\
			\hline [68 - 72) & 2.50\\
			\hline [64 - 68) & 2.75\\
			\hline [60 - 64) & 3.00\\
			\hline [0 - 60) & 5.00\\
			\hline
		\end{tabular}
	\end{center}
\section*{Tentative Course Schedule}
\begin{longtable}{|p{0.9in}|p{0.7in}|p{0.7in}|p{3in}|}
	\hline \textbf{Date} & \textbf{Day} & \textbf{} & \textbf{Topic/Activity}\\
	\hline 5 August	& Wednesday	& Lecture	 & Course Introduction\\
	\hline 7 August & Friday	& Laboratory	& Introduction to Assembly Language Programming\\
	\hline 12 August &	Wednesday &	Lecture	& Computer Organization\\
	\hline 14 August &	Friday	& Laboratory &	Registers\\
	\hline 19 August &	Wednesday &&	\textit{NO CLASS (QUEZON CITY DAY)}\\
	\hline 21 August &	Friday	&&\textit{NO CLASS (NINOY AQUINO DAY)}	\\
	\hline 26 August &	Wednesday &	Lecture &	Computer Arithmetic\\
	\hline 28 August &	Friday	& Laboratory	& Load and Store Operations\\
	\hline 2 September &	Wednesday	& Lecture &	Representing Real Numbers\\
	\hline 4 September &	Friday	& Laboratory	& Bit Operations\\
	\hline 9 September &	Wednesday	& Lecture &	Memory\\
	\hline 11 September &	Friday	& Laboratory	& Conditionals and Loops\\
	\hline 16 September	& Wednesday	& Lecture	& Memory\\
	\hline 18 September	& Friday	& Laboratory	& Functions\\
	\hline 23 September	& Wednesday	& Lecture	& Interfaces and I/O Devices\\
	\hline 25 September	& Friday	& Laboratory	& Floating Point Arithmetic\\
	\hline 26 September	& Saturday	&& \textit{FIRST WRITTEN EXAM}	\\
	\hline 1 October &	Thursday	&& \textit{MID-SEMESTER}	\\
	\hline 3 October &	Saturday	&& \textit{HANDS-ON CODING EXAM}	\\
	\hline 7 October &	Wednesday	& Lecture	& Introduction to Electric Circuits: Laws of Electricity\\
	\hline 9 October &	Friday	& Laboratory &	Introduction to Electronics\\
	\hline 14 October &	Wednesday	& Lecture &	Network Analysis\\
	\hline 16 October &	Friday	& Laboratory &	Circuits and Networks\\
	\hline 21 October &	Wednesday	& Lecture & 	Karnaugh Maps\\
	\hline 23 October &	Friday	& Laboratory &	Digital Logic Circuits\\
	\hline 28 October &	Wednesday	& Lecture &	Combinational and Sequential Logic\\
	\hline 30 October &	Friday	& Laboratory &	Complex ICs\newline
			DEADLINE OF DROPPING\\
	\hline 4 November &	Wednesday	& Lecture &	Building a Processor\\
	\hline 6 November &	Friday	& Laboratory &	Sequential Logic\\
	\hline 7 November &	Saturday &&	\textit{CIRCUITS EXAM}	\\
	\hline 11 November &	Wednesday &	Lecture &	Pipelining\\
	\hline 13 November &	Friday	& Laboratory &	Introduction to Arduino\\
	\hline 18 November &	Wednesday &	 &	\textit{NO CLASSES (APEC Meeting)}\\
	\hline 20 November &	Friday	& Laboratory &	Analog and Digital Input\\
	\hline 25 November &	Wednesday &	Lecture &	CPU Performance, Parallelism\\
	\hline 27 November &	Friday	& Laboratory &	Analog and Digital Output\\	
	\hline 28 November &	Saturday && \textit{SECOND WRITTEN EXAM}\\
	\hline 1 December &	Tuesday	&& \textit{INTEGRATION PERIOD}	\\
	\hline 2-10 December &&&		\textit{FINALS WEEK\newline (ARDUINO PROJECT DEMO)}	\\
	\hline
\end{longtable}
\section*{References}

\end{document}
